\documentclass[11pt]{article}
%\usepackage{helvet}
\usepackage{lastpage, fancyhdr,color,amsmath,amssymb,amsfonts,amscd, graphicx,latexsym,multirow}
\usepackage{makecell}
\usepackage{lipsum}
\usepackage{eurosym}
\usepackage[all]{xy}
\pagestyle{empty}
%\usepackage[turkish]{babel}
\usepackage[utf8]{inputenc}
\usepackage[T1]{fontenc}
\setlength\voffset{-1in}
\setlength\hoffset{-1in}
\setlength\topmargin{3cm}
\setlength\oddsidemargin{2.5cm}
\setlength\textheight{8in}
\setlength\textwidth{6.51in}
\setlength\footskip{1in}
\setlength\headheight{12pt}
\setlength\headsep{0.06in}
\usepackage{tikz}
%plus4mm minus3mm}
\usepackage{caption, subcaption, amsfonts}
\usepackage{setspace}
\usepackage{epstopdf}
\usepackage{relsize}
\usepackage{pgf, float}
\usepackage{url}
\newcommand{\gal}{Gal({\mathbb Q})}
\newcommand{\N}{\mathbb N}
\newcommand{\F}{\mathcal F}
\renewcommand{\H}{\mathcal H}
\newcommand{\Q}{\mathbb Q}
\newcommand{\Z}{\mathbb Z}
\newcommand{\R}{\mathbb R} 
\newcommand{\C}{\mathbb C}
\newcommand{\p}{\mathbb P}
\newcommand{\B}{\mathbb B}
\def\modorb{\mbox{$\otimes\hspace{-1.5mm}-\!\!\!-\!\!\!-\hspace{-1.5mm}\circledast$}}
\hyphenation{or-bi-fold}
%\input{defs.tex}

%\usepackage[usenames,dvipsnames]{color}
%\usepackage{epsfig}


% ’

\definecolor{light-gray}{gray}{0.55}
\newcommand\Note[1]{\textcolor{red}{{#1}}}
\begin{document}
\pagestyle{fancy}
   \lhead{}\rhead{}
   \lfoot{\textcolor{light-gray}{\small Proje Sonuç Raporu}}
        \cfoot{{\thepage}}
        \rfoot{}
  \renewcommand{\headrulewidth}{0pt}
%  \renewcommand{\footrulewidth}{0.4pt}
%\def\HG{{\bf HG-GalAct }}

\label{coverpage}

\renewcommand{\figurename}{\bf Şekil}
\renewcommand{\tablename}{\bf Tablo}
\renewcommand{\contentsname}{İçindekiler}
\renewcommand{\refname}{Referanslar}
\renewcommand{\listfigurename}{Şekiller}
\renewcommand{\listtablename}{Tablolar}
%\renewcommand{\familydefault}{\sfdefault}

\newpage
\phantom{22}
\vspace{-3cm}

% % % {\selectlanguage{turkish}\bfseries\color{red} \small Başvuru Formunun ``Bütçe ve Gerekçesi (13. Madde)'' dışındaki bölümleri toplamda Arial, 9 yazı tipinde 20 sayfayı geçmemelidir!}
% % % 
% % % \begin{center}
% % % 	\selectlanguage{turkish}\bfseries \footnotesize Kariyer proje önerisi değerlendirme formuna \\ \url{http://www.tubitak.gov.tr/tubitak_content_files/ARDEB/destek_prog/danisman_panelist/3501_DA_Panelist_Proje_Onerisi_Degerlendirme_Formu.doc} \\
% % % 	adresinden ulaşabilirsiniz.
% % % \end{center}


\begin{center}
\includegraphics[width=100px]{tubitak_logo.jpg}

\bigskip
\bigskip


\bigskip
{\fontsize{15}{10}\selectfont 
\bigskip


\bigskip
\medskip
{ \textbf{\Huge PROJE BAŞLIĞI \\}}}


\bigskip
\medskip
{ \textbf{ PROJE SONUÇ RAPORU}}
\bigskip





\end{center}


\bigskip

\bigskip


\thispagestyle{empty}


\begin{center}
\medskip
{\LARGE \textbf{Program Kodu:} 1001}

\bigskip
{\LARGE \textbf{Proje No:} 123E456}

\bigskip
{\LARGE Proje Yürütücüsü:\\
\textbf{Can Alkan}}

\end{center}

\bigskip


\bigskip
\noindent
{\large
\noindent
\underline{\bf Bursiyerler:}

\noindent
Mr. Smith

\noindent
Amy Pond

\noindent
Rory Williams

\noindent \underline{\bf Projeden desteklenmeyen diğer araştırmacılar:}

\noindent
Clara Oswald:  Destekleyen kurum çalışanı

\noindent
Rose Tyler: Bilkent Üniversitesi lisans öğrencisi (proje sırasında)


}



\bigskip





\begin{center}
{\Large TEMMUZ 2016

\vspace{1mm}
ANKARA}
\end{center}


\linespread{1.5}

\newpage\setlength{\parskip}{3mm} 
\onehalfspacing
\bigskip
\pagenumbering{roman}
\setcounter{page}{1}
\begin{center}
{\LARGE \bf ÖNSÖZ}
\end{center}
\addcontentsline{toc}{section}{ÖNSÖZ}


Proje süresince proje ekibinin şu yayınları hazırlanmıştır: 

\begin{itemize}
\item paper 1, 2014.
  
\end{itemize}

Proje süresince proje ekibinin şu bildiriler sunulmuştur:
 
\begin{itemize}
\item Sunum 1, 2015.
  
\end{itemize}

Proje kapsamında desteklenen aşağıdaki bursiyerler,  lisansüstü eğitimlerini 
proje yürütücüsünün danışmanlığında
yapmıştır.
 
\begin{itemize}
\item Amy Pond, Bilkent Üniversitesi Bilgisayar Mühendisliği, Yüksek Lisans, Ağustos 2014.
\item Rory Williams, Bilkent Üniversitesi Bilgisayar Mühendisliği, Yüksek Lisans, Aralık 2014.
\end{itemize}


\bigskip
\hfill Can Alkan

\hfill Ankara, Temmuz 2016
\newpage

\setlength{\parskip}{1mm} 

\tableofcontents
\listoffigures
\listoftables



\newpage \setlength{\parskip}{3mm}
\phantom{ss}
\vspace{-2.5cm}

\begin{center}
{\bf \Large ÖZET} 
\end{center}
\addcontentsline{toc}{section}{ÖZET}
\noindent

Bu projede $\ldots$

{\bf Anahtar kelimeler:} 1, 2, 3, 4

\newpage
\phantom{ss}
\vspace{-2.5cm}


\begin{center}
{\bf \Large ABSTRACT}
\end{center}
\addcontentsline{toc}{section}{ABSTRACT}
\noindent

In this project $\ldots$

{\bf Keywords:} 1, 2, 3, 4

\newpage
\pagenumbering{arabic}
\setcounter{page}{1}

\begin{center}
{\bf \Large 1. GİRİŞ}
\end{center}
\addcontentsline{toc}{section}{1. GİRİŞ}

\bigskip
\noindent

\lipsum[1-2]


\noindent

\clearpage
\begin{center}
{\bf \Large 2. LİTERATÜR ÖZETİ}
\end{center}
\addcontentsline{toc}{section}{2. LİTERATÜR ÖZETİ}

\lipsum[2-4]

\begin{center}
{\bf \Large 3. GEREÇ VE YÖNTEM} 
\end{center}
\addcontentsline{toc}{section}{3. GEREÇ VE YÖNTEM}
\noindent

{\bf \large 3.1. Alt başlık 1}
\addcontentsline{toc}{subsection}{3.1. Alt başlık 1}

\lipsum[4-5]

{\bf \large 3.2. Alt başlık 2}
\addcontentsline{toc}{subsection}{3.2. Alt başlık 2}

\lipsum[6-7]

{\bf 3.2.1. Daha da alt başlık }
\addcontentsline{toc}{subsubsection}{3.2.1. Daha da alt başlık}

\lipsum[8-9]

\clearpage
\begin{center}
{\bf \Large 4. BULGULAR}
\end{center}
\addcontentsline{toc}{section}{4. BULGULAR}
\noindent

\lipsum[9-10]

{\bf \large 4.1. Alt başlık}
\addcontentsline{toc}{subsection}{4.1. Alt başlık}

\lipsum[11-12]

{\bf 4.1.1. Daha da alt başlık }
\addcontentsline{toc}{subsubsection}{4.1.1. Daha da alt başlık}

\lipsum[13-14]


\clearpage

\begin{center}
{\bf \Large 5. SONUÇ}
\end{center}
\addcontentsline{toc}{section}{5. SONUÇ}

\lipsum[15-16]

\noindent
{\bf \Large 5.1. Öneriler}
\addcontentsline{toc}{section}{5.1. Öneriler}

\lipsum[17-18]



{\small 
\bibliographystyle{plain}
\bibliography{mypapers}

}




\label{endsectionb1}
\end{document}
