\documentclass[a4paper,10pt]{article}
\usepackage[utf8]{inputenc}
\usepackage{eurosym}
\usepackage{fancyhdr}
\usepackage{xspace}
\usepackage{graphicx}
\usepackage[paperwidth=18cm,paperheight=24cm,margin=1em,bottom=1.5em]{geometry}
\usepackage[a4,frame,center,noinfo]{crop}
\usepackage{helvet}
\renewcommand{\familydefault}{\sfdefault}

\title{TÜBİTAK Gelişme Raporu}
\newcommand{\hr}{\noindent\makebox[\linewidth]{\rule{\paperwidth}{0.4pt}}}

\begin{document}
\pagestyle{empty}
%\pagenumbering{gobble}
%\lfoot{ AP-GR-02 Güncelleme Tarihi 08/08/2015 } I don't think this is compulsory
\begin{center}
\vspace*{1cm}

\Huge{{\bf 
TÜBİTAK \\
ARAŞTIRMA PROJESİ\\
GELİŞME RAPORU\\ }
\textit{(Bilimsel Rapor)}\\
}

\end{center}

\vspace*{1cm}

\newcommand{\projeno}{111E111\xspace{}}

\Large{ {\bf
\begin{tabular}{lcl}
PROJE NO	&			: & \projeno \\
RAPOR NO	&			: & \\
RAPOR DÖNEMİ &		: &  01/01/2017  -  01/06/2017 \\
PROJE YÜRÜTÜCÜSÜ &	: & \\
\end{tabular}
}
}


\vspace*{1cm}
\normalsize

\begin{center}
{\bf BİLİMSEL RAPORDA OLMASI GEREKEN BİLGİLER}\\
\end{center}

\begin{enumerate}
\item Dönem içinde projeyle ilgili bilimsel ve teknik gelişmeler proje planı ile karşılaştırılarak verilmeli, elde edilen veriler ile varılan ara sonuçlar, varsa materyal, yöntem ve kapsam değişikleri belirtilmeli ve tartışılmalıdır. 

\item Dönem içindeki idari gelişmeler (yardımcı araştırıcı ve personel değişikliği, ek süre, yürütücünün kurum değişikliği ve varsa diğer destekleyen kuruluşlarla sürdürülen işbirliği, vb. konularındaki bilgiler) verilmelidir. 

\item Proje çalışmaları kabul edilen çalışma takvimine uygun yürümüyorsa gerekçeleri açıklanmalıdır. 

\item Bir sonraki dönem içinde yapılması planlanan çalışmalar (öneri formundan farklı bir durum oluşmuş ise) belirtilmelidir. 

\item Destekleyen diğer kuruluşlarla ilgili sorunlar var ise ayrıntıları ve çözüm önerileri sunulmalıdır. 
\end{enumerate}

\scriptsize

\textbf{Bilgi Notu:} \\

\begin{itemize}
\item [--] TÜBİTAK tarafından kabul edilebilir geçerli bir mazeret bildirilmeksizin; proje gelişme raporlarının sözleşmede belirtilen tarihlerde, proje sonuç raporlarının ise, sözleşmede belirtilen proje bitiş tarihinden itibaren 2 (iki) ay içinde gönderilmemesi halinde, ilgili rapor dönemine ait Proje Teşvik İkramiyeleri (PTİ) ödenmeyecektir. 


\item [--]  TÜBİTAK tarafından kabul edilebilir geçerli bir mazeret bildirilmeksizin; proje gelişme raporlarının sözleşmede belirtilen tarihlerde, proje sonuç raporlarının ise, sözleşmede belirtilen proje bitiş tarihinden itibaren 2 (iki) ay içinde gönderilmemesi halinde, ilgili rapor dönemine ait Proje Teşvik İkramiyeleri (PTİ) ödenmeyecektir. 

\item [--]  Proje ekibi tarafından, TÜBİTAK desteği ile yürütülmekte/sonuçlandırılmış olan projeler kapsamında yapılan yayınlarda [makale, kitap, bildiri (sözlü sunum/poster sunum), tez, yayılım vb.] proje sözleşmesi ve TÜBİTAK Araştırma ve Yayın Etiği Kurulu Yönetmeliği (AYEK) gereğince ilgiliproje numarası ile birlikte TÜBİTAK desteği belirtilmelidir.

\item [--]  03/11/2012 tarihinden sonra sonuçlanan projelerde, projelerin yürütücü ve araştırmacılarını ödüllendirmek amacıyla Proje Performans Ödülü (PPÖ; ppo.tubitak.gov.tr) uygulamasına başlanmıştır. Bu uygulamaya paralel olarak proje çıktılarının değerlendirilmesi de ARDEB Proje Takip Sistemi (ardeb-pts.tubitak.gov.tr) üzerinden yapılmaktadır. Bu kapsamda projenize ait çıktıların PTS'ye yüklenmesi önem taşımaktadır.	

\end{itemize}

\normalsize
\newpage
\begin{center}
\textbf{BİLİMSEL GELİŞME RAPORU EK SAYFASI}\\
(Proje No: \projeno)
\end{center}
\hr

\noindent\subsection*{1. Dönem İçinde Projeyle İlgili Bilimsel ve Teknik Gelişmeler}
\null 



\hr\noindent\subsection*{2. Dönem İçinde İdari Gelişmeler}
\null 




\hr\noindent\subsection*{3. Proje Çalışma Takvimine Uygun Yürümüyorsa Gerekçeleri}
\null 



\hr\noindent\subsection*{4. Bir Sonraki Dönemde Yapılması Planlanan Çalışmalar}
\null 




\hr\noindent\subsection*{5. Destekleyen Diğer Kuruluşlarla İlgili Sorunlar Varsa Ayrıntıları ve Çözüm Önerileri}
\null 



\hr\noindent\subsection*{6. Dönem İçinde Proje Kapsamında Yapılan veya hazırlanan Yayımlar ve Toplantılarda Sunulan Bildiriler}
\null

\normalsize
\begin{tabular}{|l|l|l|l|l|l|}
\hline
\textbf{Sıra} & \textbf{Çıktı türü} & \textbf{Yazarlar} & \textbf{Başlık} & \textbf{Yayın yeri} & \textbf{Durumu$^\ast$} \\ \hline
1             &                     &                   &                 &                     &                 \\ \hline
2             &                     &                   &                 &                     &                 \\ \hline
\end{tabular}\\

$^\ast$Hakem değerlendirmesinde, Yayınlanmaya kabul edildi, Yayınlandı.

\end{document}
